%%
%% This is file `sample-xelatex.tex',
%% generated with the docstrip utility.
%%

%% The original source files were:
%%
%% samples.dtx  (with options: `sigconf')
%% 
%% IMPORTANT NOTICE:
%% 
%% For the copyright see the source file.
%% 
%% Any modified versions of this file must be renamed
%% with new filenames distinct from sample-sigconf.tex.
%% 
%% For distribution of the original source see the terms
%% for copying and modification in the file samples.dtx.
%% 
%% This generated file may be distributed as long as the
%% original source files, as listed above, are part of the
%% same distribution. (The sources need not necessarily be
%% in the same archive or directory.)
%%
%% The first command in your LaTeX source must be the \documentclass command.
%\documentclass[sigconf]{acmart}
\documentclass[10pt,conference]{IEEEtran}
%\documentclass[sigconf,review,anonymous]{acmart}

\input{solidity}

\IEEEoverridecommandlockouts
% The preceding line is only needed to identify funding in the first footnote. If that is unneeded, please comment it out.
\usepackage{cite}
\usepackage{amsmath,amssymb,amsfonts}
\usepackage{algorithmic}
\usepackage{graphicx}
\usepackage[numbers]{natbib}
\usepackage{textcomp}
\usepackage{xcolor}
\usepackage{listings}
\usepackage{courier}
\usepackage{xurl}
\usepackage[hidelinks]{hyperref}
\usepackage{relsize,xspace}
\usepackage{fancyhdr}%construct and control page headers and footers

\usepackage{listings}
\usepackage{xcolor}

\usepackage{todonotes}
%\usepackage[disable]{todonotes}
\newcommand{\reviewer}[1]{\todo[color=orange!40, inline]{\footnotesize{Reviewer: #1 }}}
\newcommand{\henrique}[1]{\todo[color=blue!40, inline]{\footnotesize{Henrique: #1 }}}
\newcommand{\darin}[1]{\todo[color=green!40, inline]{\footnotesize{Darin: #1}}}

\newcommand{\loc}{SLoC\xspace}
\newcommand{\loctext}{Source Lines of Code\xspace}

\newcommand{\totalContracts}{26,799\xspace}

\definecolor{codegreen}{rgb}{0,0.6,0}
\definecolor{codegray}{rgb}{0.5,0.5,0.5}
\definecolor{codepurple}{rgb}{0.58,0,0.82}
\definecolor{backcolour}{rgb}{0.95,0.95,0.92}

\lstdefinestyle{mystyle}{
    backgroundcolor=\color{backcolour},   
    commentstyle=\color{codegreen},
    keywordstyle=\color{magenta},
    numberstyle=\tiny\color{codegray},
    stringstyle=\color{codepurple},
    basicstyle=\ttfamily\footnotesize,
    breakatwhitespace=false,         
    breaklines=true,                 
    captionpos=b,                    
    keepspaces=true,                 
    numbers=left,                    
    numbersep=5pt,                  
    showspaces=false,                
    showstringspaces=false,
    showtabs=false,                  
    tabsize=2
}

\lstset{style=mystyle}
%% NOTE that a single column version may be required for 
%% submission and peer review. This can be done by changing
%% the \doucmentclass[...]{acmart} in this template to 
%% \documentclass[manuscript,screen]{acmart}
%% 
%% To ensure 100% compatibility, please check the white list of
%% approved LaTeX packages to be used with the Master Article Template at
%% https://www.acm.org/publications/taps/whitelist-of-latex-packages 
%% before creating your document. The white list page provides 
%% information on how to submit additional LaTeX packages for 
%% review and adoption.
%% Fonts used in the template cannot be substituted; margin 
%% adjustments are not allowed.
%%
%%
%% \BibTeX command to typeset BibTeX logo in the docs
\AtBeginDocument{%
  \providecommand\BibTeX{{%
    \normalfont B\kern-0.5em{\scshape i\kern-0.25em b}\kern-0.8em\TeX}}}


\begin{document}

%%
%% The "title" command has an optional parameter,
%% allowing the author to define a "short title" to be used in page headers.
%\title{Solidity Smart Contracts: Language constructs for control/currency exchange and guards}
\title{An Exploratory Study on Solidity Guards and Ether Exchange Constructs}

% Double blind Workshop - No author information in this version


\author{
\IEEEauthorblockN{Darin Verheijke}
\IEEEauthorblockA{\textit{Department of Computer Science} \\
\textit{University of Antwerp}\\
Antwerp, Belgium \\
darin.verheijke@student.uantwerpen.be}
\and
\IEEEauthorblockN{Henrique Rocha}
\IEEEauthorblockA{\textit{Department of Computer Science} \\
\textit{Loyola University Maryland}\\
Baltimore, USA \\
henrique.rocha@gmail.com}
}


\maketitle

%%
%% The abstract is a short summary of the work to be presented in the
%% article.
\begin{abstract}
Ethereum is a blockchain platform that enables the use of smart contracts. Smart contracts will execute a set of instructions without an intermediary party when called upon. The possibility to make calls to another contract or exchange cryptocurrency allows for potential exploits to occur, most notable reentrancy. The Solidity language for coding smart contracts has syntactic constructs created to be safer alternatives, and guards to aid in securing code against exploits. In this paper, we collect a total of \totalContracts verified Solidity smart contracts from Etherscan, to analyze the language constructs used in calling another contract or exchanging ether. We also analyze the usage of guards to make the code more secure. For instance, even though call is the unsafest function, it is still used by 50\% of the contracts in our dataset. The safe method transfer is used by approximately one-third of contracts, and send is rarely used.  
\reviewer{abstract: to be included a final sentence about the results achieved from your study, currently not said} \henrique{added the sentences bellow, because that is the final results presented in the conclusion.}
We noticed that contracts using call have a higher average and median size in Lines of Code than normal. We also found an increased percentage of call contracts using more guards. 

%In this paper we will discuss and look at how one of these exploits, called the reentrancy attack, is possible. This attack is most well known for The DAO attack in 2016 where almost 55 million dollar got drained by an attacker who made use of this vulnerability in the smart contract. More specifically we will look at the concept in the Solidity programming language which was made specifically for the Ethereum blockchain. Also an overview of the different advantages and disadvantages of the different functions to exchange Ether, the currency used to execute transactions, will be given, how they work and how the call function might allow for certain exploits. The reentrancy vulnerability is still prevalent in smart contracts nowadays and forms a huge threat to applications and their users due to the huge possible financial losses that can happen. This research also includes an analysis of a verified smart contract database that was collected from Etherscan. This was done to detect potential vulnerabilities by looking for the presence of functions that allow for these attacks. Finally, the results of this analysis are discussed and there will be a brief discussion about prevention measures and methods to avoid a reentrancy attack.
\end{abstract}

\begin{IEEEkeywords}
Ethereum, Smart Contracts, Solidity,  Call,  Guards.
\end{IEEEkeywords}


\section{Introduction}

\reviewer{- Keywords: typically up to 5 keywords could be inserted...I'd add e.g. 'blockchain' or 'Ethereum' at least- table I could be}
\henrique{Added Ethereum to the keywords.}

A blockchain is an append-only transactional database where the information is structured together in groups, also known as blocks \cite{smart_inspect, smarter}. Each block has certain storage capacities and is chained onto the previous filled block, thus forming a blockchain. Another way to define it is a shared, immutable ledger that records transactions that can be used to track different assets. The most notable uses of this blockchain technology are the cryptocurrencies Bitcoin\cite{article} and Ether \cite{ethereum, white_paper} on the Ethereum network. One important difference between these two blockchain platforms is that Ethereum enables the deployment of smart contracts.

A smart contract is a contract that executes automatically when called upon where the terms between the two parties are written in code (on the blockchain). These contracts then run when a function is called and the conditions for that function are met and can be used to automate executions of agreements without any intermediary party \cite{criminal, 10.1145/2993600.2993611, smarter}. In its simplest form, a contract is just a collection of functions. Interesting to note is that all smart contract transactions are traceable, transparent, and also irreversible \cite{smart_inspect, smarter}.

A common functionality of smart contracts is the possibility to make calls to another contract on the same blockchain platform. This however needs to be done with caution as untrusted contracts can not only introduce errors but also risks as the contract or call may execute malicious code and exploit vulnerabilities. Every call transfers execution control to the called contract.

One of these dangers when calling an external contract is called reentrancy and is one of the most well-known attacks due to the DAO Attack in June 2016 where around 3.6 million Ether was taken which equated to around \$50 million dollar at the time \cite{10.1007/978-3-662-54455-6_8}. %This exploit is cemented in the history of Ethereum as it resulted in Ethereum being forked into Ethereum Classic and the Ethereum we know today.
The original version of this attack involved functions that would be called repeatedly before the first function was finished.

Solidity is one of the major programming languages for smart contracts on Ethereum. To avoid these exploits, there have been introduced more language constructs and recommended coding patterns. More specifically, the function call() was to be replaced by the safer functions transfer() and send(). However, recently there has been a switch back to the call() function with the introduction of EIP 1884 \cite{eip1884}. Other precautions instead must be taken to prevent reentrancy attacks, one recommendation is making use of safe code patterns and using guards.

In this paper, we conduct an exploratory research investigating how these calling functions are used in practice and if they are still commonly used for the current smart contracts being deployed in the Ethereum network. For this study, we collected a dataset of \totalContracts unique open-source verified smart contracts from Etherscan (from 2012-07-07 to 2022-01-06). We present different characteristics for the contracts and the Solidity language constructs being used.

%\henrique{Since we have space, lets add the 'in the remainder of this paper section'}
The remainder of the paper is organized as follows. In Section~\ref{sec:background}, we explain some important concepts about Solidity, its language constructs, and Reentrancy attack. In Section~\ref{sec:study-design}, we describe our dataset and our research method. Section~\ref{sec:results} presents the results from our exploratory study and a general analysis of them. Section~\ref{sec:related-work} discusses the related work. Finally, in Section~\ref{sec:conclusion}, we present our final remarks and outline future work ideas.


\section{Background}\label{sec:background}
%An introduction to some important concepts such as blockchain, the consensus mechanism used in Ethereum, smart contracts and reentrancy attacks is given. The scope of these concepts will be kept to the Ethereum blockchain and one of its programming languages Solidity \cite{solidity}.


%\subsection{Ethereum \& Smart contracts}
%Ethereum differs from Bitcoin in that it enables the deployment of smart contracts and decentralized applications also known as dApps with built-in economic functions. While bitcoin its primary focus is to be a digital currency payment network, Ethereum is designed to be a general-purpose programmable blockchain that runs a virtual machine capable of executing code.
%Ether is the currency used to complete transactions on the network and is used as a way to meter and constrain execution resource costs. In comparison to bitcoin, Ether is designed to be a utility currency which is used to call transactions on the Ethereum platform as a sort of fee \cite{mastering}. 
%Smart contracts can be used to create a range of dApps.  An important feature of smart contracts is that when a function is called and the conditions for that function are met there is an automatic execution of the set agreements without any intermediary party \cite{ethereum, white_paper}.


\subsection{Solidity}

Solidity~\cite{solidity} is one of the main languages to code smart contracts in the Ethereum platform. Solidity is an object-oriented, high-level language that has syntax comparable to C++.

\subsubsection{Solidity Ether Exchange}

We like to highlight the language constructs used to exchange cryptocurrency among contracts: call, send, and transfer (Table~\ref{tab:freq}).

\begin{table}[h]
\center
  \caption{Solidity Functions to Exchange Ether}
  \label{tab:freq}
  \begin{tabular}{ccl}
    \hline
    Function & Gas Limit & Error Handling\\
    \hline
    call & Custom & Returns false on failure\\
    transfer & 2300 & Throws exception on failure\\
    send & 2300 & Returns false on failure\\
  \hline
\end{tabular}
\end{table}

\reviewer{table I could be moved in the middle (not at the beginning) of page 2, after being cited in the text of Section II}
\henrique{I used the [h] ("here")  it forces tex to display the table/figure where it is instead of trying to put on top of a column. }

The call function is a low-level interface for sending a message to a contract and it is also a way to send Ether to another address. The call function transfers the execution control to the called contract and the caller can forward any amount of gas. Therefore, the call function has the potential to introduce vulnerabilities, most notably reentrancy.

The transfer method was first introduced in version 0.4.10 (May 2017) of the Solidity language. It provides a safe-by-design method to transfer cryptocurrency. Even though this method also transfers the execution control to the caller, it has a gas limit that prevents abuse. If the transfer fails, an exception is raised, which also adds to the security of this method as the exception reverts the transaction. Due to automatically reverting in case of errors, the transfer function is recommended in most cases.

The send function can be seen as a lower-level implementation of transfer. Similar to transfer, it provides a safe-by-design function to transfer cryptocurrency, with a gas limit to prevent exploits. The major difference between send and transfer, is that send returns false if it fails, delegating the error handling to the developer.


\subsubsection{Solidity Guards}

Guards are language constructs to prevent access or revert a transaction. In Solidity, \textit{Require} and \textit{Assert} have been introduced to the language in the version 0.4.10 (May 2017); \textit{Revert} was introduced in version 0.4.12 (Ago 2017).

Both {require} and {assert}, check for a condition and raise an exception if such condition is not met. Any exception in a smart contract execution will cancel the transaction. Assert is supposed to be a check for internal errors and bugs. Assert will consume all remaining gas. On the other hand, require intent is to be used as much as possible for developers to check for conditions. Require refunds remaining gas if it raises an exception.

Revert raises an exception while refunding the remaining gas. It is similar to a "throws new Exception()" in Java.

Those three methods (assert, require, and revert) are guards to stop a transaction and prevent possible exploits in a smart contract. The usage of these constructs may indicate that developers are concerned with the security of the contract.

\subsection{Reentrancy attack}

The call function has some vulnerabilities. Every call to another contract transfers execution control to the called contract. Untrusted contracts may introduce and execute malicious code or exploit vulnerabilities. One of these major vulnerabilities is called the reentrancy attack, which takes advantage of the transfer of execution control by making recursive calls back to the original contract, repeating executions, and creating new transactions.

The two main types of reentrancy attacks are Single function, and Cross-function.

\subsubsection{Single function reentrancy attack}
This version repeatedly calls the involved function before the first invocation of the function is finished.  Listing~\ref{lst:reentrancy1} shows a code snipped with this exploit.

\begin{lstlisting}[language=Solidity, caption=Single function reentrancy attack, label=lst:reentrancy1]
mapping (address => uint) private userBalances;

function withdrawBalance() public {
  uint amountToWithdraw = userBalances[msg.sender];
  (bool succes, ) = msg.sender.call.value(amountToWithdraw)("");
  require(success);
  userBalances[msg.sender] = 0;
}
// Fallback function which gets executed
function () public payable {
  withdrawBalance()
}
\end{lstlisting}

In this example,  an attacker can recursively call the \texttt{withdrawBalance()} function and drain the whole contract as the user's balance is only set to 0 at the very end of the function.

\subsubsection{Cross-function reentrancy attack}
When a function shares a state with another function there is a possibility of a cross-function reentrancy attack.  Listing~\ref{lst:reentrancy2}  shows a code snipped with a cross-function reentrancy vulnerability. 

\begin{lstlisting}[language=Solidity, caption=Cross-function reentrancy attack, label=lst:reentrancy2]
mapping (address => uint) private userBalances;

function transfer(address to, uint amount) {
  if(userBalances[msg.sender] >= amount) {
     userBalances[to] += amount;
     userBalances[msg.sender] -= amount;
  }
}
function withdrawBalance() public {
  uint amountToWithdraw = userBalances[msg.sender];
  (bool succes, ) = msg.sender.call.value(amountToWithdraw)("");
  require(success);
  userBalances[msg.sender] = 0;
}

\end{lstlisting}

Here the attacker will call the transfer function when the code is executed on an external call in \texttt{withdrawBalance()},  again the user's balance is not yet set to 0 and thus they will be able to transfer tokens again. A simple solution to both these types of attacks is updating the balance before transferring control to another function or contract.  Another simple solution would be to use transfer or send (the safer-by-design constructs) instead of call.


\section{Study Design}\label{sec:study-design}
\subsection{Dataset}

We collected verified smart contracts from Etherscan\footnote{\url{etherscan.io/contractsVerified}} which is a block explorer and analytic platform for Ethereum. Etherscan verified contracts allows the public to audit and read contracts as it has to be made publicly available to be granted the verified status. Etherscan does not give access to a complete dataset of verified smart contracts but rather has an open-source database of the latest 10,000-5,000 smart contracts that were verified. Therefore, we gathered the latest contracts from time to time, over a period of six months (2021-07-07 to 2022-01-06) to build our dataset.

Then, we did the following pre-processing steps in our dataset: (i) remove all duplicated\footnote{We removed contracts with the same address in the Ethereum blockchain. We did not verify whether the contracts with the same name have the same source code. Since these contracts have different addresses, they are considered separate entities in the blockchain platform.} contracts; (ii) remove contracts not written in Solidity; (iii) removed contracts which we could not process using cloc.\footnote{cloc is a tool to count source lines of code available at <\url{https://github.com/AlDanial/cloc}>.} After removing those contracts, we had a total of \totalContracts unique verified solidity smart contracts. This dataset is publicly available.\footnote{\url{https://bit.ly/3fyOgBD}}


\subsection{Method}

We use the Etherscan API~\cite{etherscan_api} to retrieve the source codes for each contract in our dataset. Listing~\ref{lst:api} shows an example of the API call used to acquire the contract source code.
\begin{lstlisting}[language=php, caption=Etherscan API call, label={lst:api}]
api.etherscan.io/api?module=contract
&action=getsourcecode
&address=0xb4e32b964f6ae78 //The contract address
&apikey=YourApiKeyToken // Your API key
\end{lstlisting}

Then, we used the cloc tool on the contracts to discover how many source lines of code\footnote{In this paper, we always use Source Lines of Code (SLoC), because that is how the tool we employed counts it. However, during the paper, we may refer to it as just lines of code (LoC) for brevity.} are in each one. We removed from this study the contracts that cloc were not able to process. Moreover, we used a Python script on the contracts to locate specific methods used in the contracts related to Ether exchange functions (call, send, and transfer) and guards (require, assert, revert).

\reviewer{Why did you chose only LOC metric? A comparison among other similar metrics has been made ? Could you discuss the choice?}
\henrique{Explained in the new paragraph bellow.}
We chose to focus on lines of code in this paper because it is a well-known metric that is simple to verify and analyze. Since this is the first exploratory study we conducted in Solidity source code, we decided to limit the scope of the research. 
Although we plan to expand this study with other metrics, we left it for future work.

\section{Analysis and Results}\label{sec:results}

\subsection{Overall Analysis}

First, we show some general characteristics of our dataset.
Table~\ref{tab:loc} shows the general statistics considering the lines of code on the contract. We can see that the contracts in our dataset are small in lines of codes, with a median of 256 LoC and an average of 356 LoC. That is expected, as smart contract code tends to be smaller when compared to software code in other domains.  The smallest contracts have only 2 LoC.  For example, the contract \textit{BlackHole}\footnote{\url{https://etherscan.io/address/0x727E9A3067DeEaF031916fA0fC53B02cf44F8731\#code}} only has two lines, a pragma definition for the solidity version, and an empty contract definition. The biggest contract is \textit{RewardControl}\footnote{\url{https://etherscan.io/address/0xcf8Fe5bB819359Ea02DF65E50B6194D12b69aB88\#code}} with 6,461 LoC. 

\begin{table}
\center
  \caption{Source Lines of Code (LoC)}
  \label{tab:loc}
  \begin{tabular}{c c c c c}
    \hline
    Min & Median & Average & Std. Dev. & Max \\
    \hline
   2 & 256 & 359 & 335 & 6,461 \\
  \hline
\end{tabular}
\end{table}

\reviewer{Table II: even if it'd be trivial, please could you add a short note/text about which LOC did you consider (SLOC or all the LOCs, included potential comments?) or better specify the acronym (e.g. SLOC and not LOC)?} 
\henrique{I added a footnote but if we change to SLoc it needs to change in ENTIRE paper and not just this table. }

Now, we investigate how many Ether exchange methods and guards are being used in the contracts. Table~\ref{tab:results-all} shows how many contracts in our dataset have at least one of the methods, the overall count of the method, and the average and median number in the contracts that have at least one.  Even though call is the unsafest method for Ether exchange, it is used by 50\% of the contracts in the dataset. On average, there are 2.3 call uses on the contracts. Any contract using call has the potential to have a reentrancy vulnerability. On the other hand, send is a safer method and it is the least used (2\%). Transfer is the safest method, and it is used by roughly one-third of contracts.

\begin{table}
\center
  \caption{Ether Exchange and Guards Usage}
  \label{tab:results-all}
  \begin{tabular}{crrrr}
    \hline
    Method & Contracts & Count & Average & Median \\
    \hline
    Call & 13,443 (50\%) & 32,236 & 2.3 & 1.0\\
    Transfer & 9,176 (34\%) & 17,814 & 1.9 & 0\\
    Send & 647 (02\%) & 1,059& 1.6 & 0\\
    Require & 26,190 (97\%) & 622,679 & 23.7 & 18.0 \\
    Assert & 7,279 (27\%) & 10,507 & 1.4 & 0\\
    Revert & 13,819 (51\%) & 41,502 & 3.0 & 1.0\\
    \hline
\end{tabular}
\end{table}

\henrique{Median values are incorrect. You are taking the median over all contracts. You should take the median only if the contract has at least one of the method. }

On the guard methods shown in Table~\ref{tab:results-all}, require is used by the great majority of all contracts (97\%). On average, there are 23.7 uses of require in the contracts. The great usage of this guard may be to counteract the vulnerabilities of call. Revert is also commonly used by more than half of the contracts in our dataset. Finally, Assert is the least used guard in our analysis.

\subsection{Contracts by Version}

Table~\ref{tab:major-versions} shows the contracts categorized by their major Solidity version, and the average and median size in lines of code (LoC). Most contracts are from the latest Solidity version, 0.8.x. The oldest Solidity version to appear in our dataset is 0.4.x which has fewer amount of contracts. The biggest average and median LoC size are from contracts of version 0.6.x.

\begin{table}
\center
  \caption{Solidity Major Versions}
  \label{tab:major-versions}
  \begin{tabular}{crrrr}
    \hline
    Version & Contracts & Average LoC & Median LoC \\
    \hline
    0.8.x & 14,869 (55\%) & 355 & 285 \\
    0.7.x & 2,454 (09\%) & 432 & 321 \\
    0.6.x & 5,838 (21\%) & 433 & 330 \\
    0.5.x & 2,954 (11\%) & 200 & 121 \\
    0.4.x & 684  (02\%) & 225 & 108 \\
  \hline
\end{tabular}
\end{table}

Figure~\ref{fig:minor-versions} shows the top-10 solidity versions (major and minor) in our contracts. The versions with most contracts are 0.8.4 (18.7\%), 0.6.12 (17.1\%), and 0.8.7 (15.3\%). Together these three versions compose over 50\% of the contracts in our dataset.

\begin{figure}[h]
  \centering
  \includegraphics[width=\linewidth]{./img/clean_versions_final.png}
  \caption{Top-10 Solidity versions in the database.}
  %\description{Solidity Compiler Version}
  \label{fig:minor-versions}
\end{figure}

\reviewer{Fig1/2/3: could you add the labels for the X and Y axis (missing)?}

\begin{table}
\center
  \caption{Contracts using Ether Exchange \& Guards by Major Version}
  \label{tab:results-version}
  \begin{tabular}{crrrrrr}
    \hline
      & Call & Send & Transfer & Require & Assert & Revert \\
    \hline
    all & 50\% & 2\% & 34\% &  97\% & 27\% & 51\% \\
    0.8.x & 44\% & 3\% & 33\%  & 98\% & 8\% & 45\% \\
    0.7.x & 67\% & 1\% & 42\% & 98\% & 32\% & 66\% \\ 
    0.6.x & 79\% & <1\% & 32\% & 98\% & 68\% & 79\% \\
    0.5.x & 16\% & <1\% & 29\% & 96\% & 32\% & 20\% \\
    0.4.x & 8\% & 3\%  & 48\% & 92\% & 51\% & 39\% \\
    \hline
\end{tabular}
\end{table}

In Table~\ref{tab:results-version}, we assess the percentage of contracts using at least one of the methods (Call, Send, Transfer, Require, Assert, and Revert) per major version of Solidity. We also included the percentage considering all contracts for comparison.  As we can see, the major version appears to have an impact on the usage of call, transfer, assert, and revert. For instance, versions 0.7.x and 0.6.x have a higher percentage of contracts using call than the normal, and versions 0.5.x and 0.4.x have a lower percentage.  Considering the transfer method,  version 0.7.x and 0.4.x show a higher percentage of contracts using transfer.  

For the guards in Table~\ref{tab:results-version},  versions 0.8.x and 0.5.x show lower percentages for asserts and reverts.  On the other hand, versions 0.7.x and 0.6.x show a higher percentage than normal for the same guard methods. Version 0.4.x shows an increase in contracts using assert but a decrease in contracts using revert. 


\subsection{Contracts using Call}

We focus only on the contracts that contain a call function. Since call is a very unsafe function, there is the possibility for the contracts using it to suffer from a reentrancy vulnerability. For this reason, such contracts may have different characteristics. Table~\ref{tab:call-loc} shows the lines of code considering only the contracts that contain a call function. The average and median LoC values are higher than the ones for all contracts. Therefore, the contract with call in our dataset usually has more lines of code. The call contract with the least lines of code is called \textit{FlashBotLowGas}.\footnote{\url{https://etherscan.io/address/0x90ab9a926a1593992547e0f9a0df6401f10421cd\#code}}

\begin{table}[h]
\center
  \caption{Contracts using Call - Lines of Code}
  \label{tab:call-loc}
  \begin{tabular}{c c c c c}
    \hline
    Min & Median & Average & Std. Dev. & Max \\
    \hline
   6 & 530 & 511 & 367 & 5,572 \\
  \hline
\end{tabular}
\end{table}

Figure~\ref{fig:call_version} shows the Solidity versions for contracts with call. The version with most contracts using call is 0.6.12.  The top-3 versions for all contracts are also the top-3 for contracts with call but in different positions. 

\begin{figure}[h]
  \centering
  \includegraphics[width=\linewidth]{./img/call_clean_final.png}
  \caption{Top-10 Solidity versions on Contracts using Call. }
  %\description{Solidity Compiler Version}
  \label{fig:call_version}
\end{figure}

Table~\ref{tab:call} show the Ether exchange and guard methods considering only the 13,443 contracts that contain at least one call method. We can see that approximately one-third of the contracts using call also use the send function. We also like to highlight that there is a greater number of contracts with call-using guards. For instance, 99\% of the call contracts used Require compared to 97\% of all contracts; 39\% of the call contracts used Assert compared to 27\% of all contracts; and 89\% of the call contracts used Revert compared to 51\% of all contracts. The higher usage of guards is probably to counter the vulnerabilities of call. This may be an indication that Solidity developers are concerned about the security of their contracts especially when using unsafe methods such as call.

\begin{table}
\center
  \caption{Solidity Methods of Contracts using Call}
  \label{tab:call}
  \begin{tabular}{crrrr}
    \hline
    Method & Contracts & Count & Average & Median\\
    \hline
    Transfer & 4,514 (33\%) & 8,789 & 0.65 & 0\\
    Send &582 (04\%) & 920 & 0.07 & 2.0\\
    Require &13,392 (99\%) & 456,461 & 33.96 & 31.0 \\
    Assert & 5,348 (39\%) & 6,701 & 0.49 & 0.0\\
    Revert & 11,976 (89\%) & 38,273 & 2.84 & 2.0\\
\end{tabular}
\end{table}

\subsection{Contracts using Transfer}

Now, we focus only on the contracts that contain a transfer function. Transfer is a much safer alternative than call for Ether exchange. Therefore, we expect the contracts using transfer to have different characteristics than the ones using call.

Table~\ref{tab:transfer-loc} shows the lines of code only from contracts with a transfer function. The median, average, and standard deviation are higher than the ones when considering all contracts. However, the same statistics are lower when compared to the contracts with call. This means that contracts using transfer tend to have lower LoC than the contracts using call. The reason for this difference could be that call will need extra code to protect against vulnerabilities. Since transfer is a safe-by-design function, it will not need as much extra code as call for a more secure contract. The smallest LoC contract using transfer is \textit{TransferValueToMinerCoinbase}\footnote{\url{https://etherscan.io/address/0x8512a66d249e3b51000b772047c8545ad010f27c\#code}} with 6 LoC.

\begin{table}[t]
\center
  \caption{Contracts using Transfer - Lines of Code}
  \label{tab:transfer-loc}
  \begin{tabular}{c c c c c}
    \hline
    Min & Median & Average & Std. Dev. & Max \\
    \hline
   6 & 345 & 452 & 373 & 6,461 \\
  \hline
\end{tabular}
\end{table}

Figure~\ref{fig:transfer_version} shows the solidity versions for contracts that contain transfer.  The top-5 versions for contracts using send are the same top-5 versions for all contracts in our dataset.  This could indicate that the contracts with send may represent a general set similar to all contracts in our dataset.

\begin{figure}[h]
  \centering
  \includegraphics[width=\linewidth]{./img/transfers_clean_final.png}
  \caption{Top-10 Solidity versions on Contracts using Transfer.}
  %\description{Solidity Compiler Version (Sends)}
  \label{fig:transfer_version}
\end{figure}

\begin{table}
\center
  \caption{Solidity Methods of Contracts using Transfer}
  \label{tab:transfer}
  \begin{tabular}{crrrr}
    \hline
    Methods & Contracts & Count & Average & Median \\
    \hline
    Call & 4,514 (49\%) & 11,083 & 1.20 & 0\\
    Send &107 (01\%) & 251 & 0.03 & 0\\
    Require & 9,060 (98\%) &256,835 & 27.99 & 23.0\\
    Assert & 2,722 (29\%) & 4,789 & 0.52 & 0\\
    Revert & 4,876 (53\%) & 14,493 & 1.57 & 1.0\\
    \hline
\end{tabular}
\end{table}


Table~\ref{tab:transfer} shows the Ether exchange and guard methods considering only the 9,176 contracts that contain at least one transfer method. The percentage of contracts when looking at only contracts with transfer is similar (with a 1-2\% difference) from the ones considering all contracts. This is different from the contracts with call where the guards' percentage increases by a noticeable amount for Assert and Revert.


\subsection{Contracts with Send}

We analyze only the contracts using at least one send method. In our dataset, send is the least used method, being present in only 647 (approximately 2\%) of the contracts. Even though there are fewer contracts to analyze, we expect to obverse different characteristics.

Table~\ref{tab:send-loc} shows lines of code for contracts with send. The median, average, and standard deviation are the highest when compared to all contracts, contracts using call, and contracts using transfer. 
The contract \textit{PaymentManager}\footnote{\url{https://etherscan.io/address/0xaddeb5dbdc1c62c2a2a8e04fddd42e3c3f19587b\#code}} has 2 send functions and it is smallest contract with 15 LoC.

\begin{table}
\center
  \caption{Contracts using Send - Lines of Code}
  \label{tab:send-loc}
  \begin{tabular}{c c c c c}
    \hline
    Min & Median & Average & Std. Dev. & Max \\
    \hline
   15 & 575 & 635 & 498 & 6,461 \\
  \hline
\end{tabular}
\end{table}

The most common Solidity versions for contracts using send were 0.8.7,  0.8.4,  0.8.0,  0.8.10,  and 0.8.6. As we can see,  the top-5 versions are all 0.8.x. 

%\begin{figure}[h]
%  \centering
%  \includegraphics[width=\linewidth]{img/transfer_versions_clean.png}
%  \caption{Top-10 Solidity versions on Contracts using Send.}
%  \label{fig:send_version}
%\end{figure}

Table~\ref{tab:send} show the Ether exchange and guard methods considering only the 647 contracts that contain at least one send function. The contracts percentage are different when contrasted with all contracts. For instance, 89\% of send contracts have a call function compared to 50\% of all contracts; 90\% of send contracts have at least one revert compared to 51\% of all contracts; and in the opposite direction, 16\% of send contracts have at least one transfer method compared to 34\% of all contracts.

\begin{table}
\center
  \caption{Solidity Methods of Contracts using Send}
  \label{tab:send}
  \begin{tabular}{crrrr}
    \hline
    Method & Contracts & Count & Average \\
    \hline
    Call& 582 (89\%) & 1,203 & 1.86 & 2.0\\
    Transfer& 107 (16\%) & 234 & 0.36 & 0\\
    Require& 647 (100\%) & 25,058 & 38.73 & 37.0\\
    Assert& 63 (09\%) & 166 & 0.26 & 0\\
    Revert& 588 (90\%) & 2304 & 3.56 & 4.0\\
    \hline
\end{tabular}
\end{table}


\section{Related work}\label{sec:related-work}


Juels, Kosba and Shi~\cite{criminal} investigate the risk of smart contracts fueling new criminal ecosystems. They show how a Criminal Smart Contract can facilitate leakage of confidential information, theft of cryptographic keys, and more, showing the urgency of creating safeguards against these CSCs. They look at questions like how practical these new crimes will be, whether these CSCs enable a wider range of new crimes in comparison to earlier cryptocurrencies such as Bitcoin, and what advantages they offer to criminals in comparison with the conventional online systems.


Luu et al.~\cite{smarter} also investigate and introduce several security problems to manipulate smart contracts in an attempt to gain profit and propose ways to enhance the operational semantics of Ethereum. A focus is put on the semantic gap between the assumption contract writers make about the underlying execution semantics and the actual semantics of the contract are made as a reason for these security flaws. A tool OYENTE is also provided to detect bugs which is a symbolic execution tool. The model works directly with Ethereum virtual machine bytecode and thus does not need a higher level representation such as Solidity. An evaluation of OYENTE on 19,366 smart contracts is given where 8,333 contracts were documented as potentially having bugs.

Mense and Flatscher~\cite{security} summarize known vulnerabilities found by literature research and analysis such as external calls, gasless sends, mishandled exceptions, and reentrancy. They also compare code analysis tools for their ability to identify vulnerabilities in smart contracts based on a taxonomy for vulnerabilities. The results of their paper show that reentrancy ranks the highest among the vulnerabilities that they have discussed and is detected by most of the tools used. They then delve deeper into the DAO hack as well.


Liu et al.~\cite{reguard} present ReGuard which is a fuzzing-based analyzer to automatically detect reentrancy bugs in Ethereum smart contracts. They iteratively generate random (but diverse) transactions, this is called fuzz testing. Then based on the runtime they will identify reentrancy vulnerabilities in a contract. How the architecture works is they parse a smart contracts source or binary code to an intermediate representation which will then be transformed to C++, keeping the original behavior. Together with a runtime library, ReGuard executes the contract and runs an analysis of the operations for any reentrancy attacks.


SmartCheck~\cite{smartcheck} is an extensible static analysis tool to detect code issues in Solidity, where it translates Solidity into an XML-based representation and checks it against XPath patterns. The authors used a real-world dataset to evaluate their tool and also make a comparison to the earlier mentioned Oyente.

Samreen and Alalfi~\cite{survey} explain eight vulnerabilities by looking at past exploitation case scenarios and reviewing some of the available tools and applications to detect these vulnerabilities. For each case they discuss the vulnerability exploited, the tactic used as well as the financial loss that happened. Coverage is given of some preventive techniques as protection against some of these exploits. The tools/frameworks they discuss adopt either a form of static analysis such as symbolic execution and control flow graph construction or dynamic analysis such as the fuzzing testing or tracing the sequence of instructions that are executed at run time.

%Tantikul and Ngamsuriyaroj~\cite{icissp20} investigate a more recent state of the vulnerabilities of smart contracts. Their research consists of going through a database of verified smart contracts and checking common occurrences as well as trends of vulnerabilities. An analysis is done using both Oyente and Smartcheck and common characteristics of vulnerable smart contracts are identified. A correlation computation is done via Pearson's correlation to detect how often any pair of vulnerabilities will be found on the same smart contract. Their results show that overflow and underflow have the highest correlation. Another relation found is the timestamp dependency and transaction order which might be caused by malicious miners.

Bragagnolo et al.~\cite{rocha} address the lack of inspectability of a deployed smart contract. They do this by analyzing the state of the contract using different decompilation techniques. Their solution SmartInspect is an inspector based on pluggable property reflection. Their approach of utilizing mirrors generated from an analysis of Solidity source code allows access to unstructured information from a deployed smart contract in a structured way. This can be done without a need to redeploy or develop additional code for decoding.

Wang et al.~\cite{contractward} evaluate a set of real-world smart contracts with ContractWard which uses machine learning techniques to detect vulnerabilities in smart contracts. Their idea was proposed due to existing detection methods being mainly based on symbolic execution or analysis which are very time-consuming. The system extracts dimensional bigram features from simplified operation codes to construct a feature space and can get a predictive recall and precision of over 96\% based on their dataset of 49502 smart contracts on 6 vulnerabilities.

A deep-learning-based approach is used by Qian et al.~\cite{automated}. The aim is to precisely detect reentrancy bugs using a bidirectional long-short term memory with an attention mechanism. They also propose using a contract snippet as another way to represent a smart contract only capturing key semantic sentences which contain related and critical information such as control flow and data dependencies. These are then used as input to the sequential models. They show that this deep-learning approach outperforms other state-of-the-art smart contract vulnerability tools.

%Slither by Feist et al.~\cite{slither} is a static analysis framework that converts Solidity smart contracts into an intermediate representation which they call SlithIR. Static Single Assignment forms are used as well as a reduced instruction set for ease of implementation. Their framework has use cases in automated detection of vulnerabilities, detection of code optimization opportunities, improvement of clarity, and ease of understanding of the contracts. An evaluation of the capabilities of the proposed framework is done using a set of real-world smart contracts.

\section{Final Remarks}\label{sec:conclusion}

In this paper, we conducted an exploratory study on the usage of specific Solidity language constructs in a dataset of \totalContracts contracts. Even though, call is the unsafest method for Ether exchange it is the most popular method being used by 50\% of contracts. Perhaps because call can be used to transfer the execution control to another contract, and not only for ether exchange, is the reason for its popularity despite the fact that it is unsafe. The other methods for ether exchange, transfer is used by 34\% of contracts, and send is rarely used (2\% of the contracts).

The most popular Solidity versions in our dataset were 0.8.4 (18.7\% of the contracts), 0.6.12 (17.1\% of the contracts), and 0.8.7 (15.3\% of the contracts). We also saw that the usage of call, transfer, assert, and revert can vary a lot from different versions of the contract.

When we focused on only contracts using call, the average and median size in LoC of the contracts are higher than normal. We also noticed an increased percentage of call contracts using more guard methods.

As future work, we plan to execute vulnerability detection tools to further investigate the characteristics of the contracts in our dataset.  We also plan to investigate different metrics besides Lines of Code.

\reviewer{- please complete the description with missing info in some references. E.g. [3] presents author, title and year of publication but is it a paper, a book or what ever else? Please insert missing details
- please, when available, insert URLs for accessing directly sources online. E.g. [3] is available at https://bitcoin.org/bitcoin.pdf.}

%%
%% The next two lines define the bibliography style to be used, and
%% the bibliography file.

\bibliographystyle{IEEEtranN}
\bibliography{references}

%\endinput
\end{document}

